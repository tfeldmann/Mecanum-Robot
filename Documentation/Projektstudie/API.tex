% !TEX root = Projektstudie.tex
% API

\section{API}
Die Kommunikation mit dem Roboter findet über die UART-Schnittstelle bei einer Baudrate von 115200 Mbaud statt.
Auf dem Arduino sorgt die Library \emph{SerialCommand}\footnote{https://github.com/kroimon/Arduino-SerialCommand} für die Interpretation der Ergebnisse.

\subsection{Befehle}
Folgende Befehle sind verfügbar:
\begin{description}
\item \lstinline{@start} \\
Initialisiert und startet die Schrittmotor-Treiberkarten gemäß Kapitel~\ref{sec:Treiber}.
Der Roboter ist daraufhin fahrbereit.

\item \lstinline{@stop} \\
Führt einen Not-Stopp durch. Dabei wird der in Kapitel~\ref{sec:Treiber} beschriebene Quickstop ausgeführt.
Anschließend befinden sich die Schrittmotor-Treiberkarten im Ruhezustand und müssen erst wieder mit dem Befehl \lstinline{@start} gestartet werden.

\item \lstinline{@v v1 v2 v3 v4} \\
Über diesen Befehl werden die Soll-Radgeschwindigkeiten vorgegeben.

Anstelle von \lstinline{v1, v2, v3} und \lstinline{v4} sind durch Leerzeichen getrennte, ASCII-codierte Integer im Bereich von $-25000$ bis $25000$ (Schritte pro Sekunde) anzugeben.
\end{description}


\subsection{Fehlercodes}
Tritt bei der Kommunikation mit der API ein Fehler auf, führt der Roboter sofort einen Not-Stopp aus und antwortet mit einem Fehlercode sowie einer Fehlerbeschreibung.
Damit ist auch bei einer Störung der Kommunikation gewährleistet, dass der Roboter nicht außer Kontrolle gerät.
Mögliche Fehlercodes sind beispielsweise:

\lstinline{!E01: Not enough arguments supplied.}\\
\lstinline{!E02: Unknown command: [command]}\\


\subsection{Ereignisse}
Um externer Software Rückschlüsse auf den Zustand des Roboters zu ermöglichen, sendet dieser bei erfolgreichen Initialisierungen oder Befehlen Ereignisprotokolle.

Fest definierte, für die steuernde Software relevante Aktivitäten werden mit vorangestelltem Index gesendet:\\
\lstinline{@A01: All motors are ready.}\\
\lstinline{@A02: New motor speed set.}\\
\lstinline{@A03: QuickStop successful.}

Sonstige Logging-Nachrichten werden mit einer Raute versehen.
Dazu zählen aktuelle Soll-/Ist-Werte und Informationen über die Firmware, wie beispielsweise\\
\lstinline{# Left: 120, Right: 230}\\
\lstinline{# Firmware Version 1.0, compiled on Tue Jun  4 18:15:59 2013}
