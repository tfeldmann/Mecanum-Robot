% !TEX root = Projektstudie.tex
% Hardware

\section{Hardwaremodifikation}
\label{sec:Hardwaremodifikation}

Um eine sichere Steuerung zu gewährleisten ist es notwendig die Hardware zu optimieren.
\begin{itemize}
\item{Die Netzanschlussleitung der vier Netzteile wird komplett erneuert. Der Schutzleiter wird angeschlossen wobei auf eine korrekte Anwendung der Zugentlastungen geachtet wird.}
\item{Die Erdung des Schutzleiters wird über Erdungsklemmen auf das Fahrzeuggehäuse gelegt.}

\item{Die verwendete Schraubenlänge zur Befestigung der Mecanum-Räder war ungenügend. Alle vier Mecanum - Räder sind nun mit je drei M5 Schrauben befestigt.}
\end{itemize}

Vor der Hardwareänderung sind je zwei Motortreiberkarten über einen separaten CAN-Bus von der verbauten SPS angesprochen worden. 
\begin{itemize}
\item{Um einen Betrieb sowohl über den Arduino als auch über die SPS zu ermöglichen werden alle Komponenten auf einen CAN-Bus zusammengelegt, die Motortreiberkarten umadressiert und die 120 Ohm Abschlusswiderstände in den äußersten Steckern eingelötet.}
\item{Da die Steckerbelegung des Arduino-CAN-Shield von der ISO 11 898 Norm abweicht, wurde der entsprechende Stecker der Busleitung angepasst.}
\item{Ein weiterer Stecker zur Diagnose der CAN-Bus-Signale via PC wird angelötet.}

\end{itemize}