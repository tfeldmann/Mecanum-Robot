% !TEX root = Projektstudie.tex
% Ausblick

\section{Ausblick}
\label{sec:Ausblick}
% - NOT-AUS Schalter
% - Arduino-Shield
% - Deckenkamera
% - Hindernissen ausweichen (Dijkstra?)
% - Steuerung übers Netzwerk / W-Lan: Python Flask Server + Raspberry

\subsection{Schutzbeschaltung}
\label{sec:Schutzbeschaltung}
%NOT-AUS



\subsection{Arduino-Shield}
\label{sec:Arduino-Shield}
Da der Speicher des verwendeten Arduino UNO bereits zu $78\%$ ausgelastet ist, wird in Ergänzung zum Sparkfun CAN-Shield ein eigenes CAN-Shield für den leistungsfähigeren Arduino MEGA 2560 entwickelt.
 
Durch geschickte Wahl der Anschlusspins wird die optional angedachte Verwendung des Arduino-Ethernet-Shield ermöglicht. Hierzu werden der Ethernet-Shield und der entwickelte CAN-Bus-Shield auf den Arduino gestapelt aufgesteckt. Neben CAN-Bus-Stecker und –Buchse mit erforderlicher Elektronikbeschaltung besitzt das Layout Anschlussklemmen für $5V$ Spannungsversorgung und $I^{2}C$-Bus.

Das Platinenlayout ist als Double Layer Platine ausgeführt. In der folgenden Abbildung
sind die Abmessungen der Bauteile  in Grau, Lötpads in Grün, die Top Kupferflächen und Leiterbahnen in Rot und die Kupferflächen  und Leiterbahnen auf der Bottom Seite in Blau dargestellt.

\begin{figure}[H]
\centering
 \includegraphics[width=0.8\textwidth]{Abbildungen/CAN-Shield-Layout} 
\caption[CAN-Shield-Layout]{CAN-Shield-Layout}
\label{fig:CAN-Shield-Layout}
\end{figure}

\begin{figure}[H]
\centering
 \includegraphics[width=0.8\textwidth]{Abbildungen/CAN-Shield-Layout-Top} 
\caption[CAN-Shield-Top-Layer]{CAN-Shield-Top-Layer}
\label{fig:CAN-Shield-Layout-Top}
\end{figure}

\begin{figure}[H]
\centering
 \includegraphics[width=0.8\textwidth]{Abbildungen/CAN-Shield-Layout-Bottom} 
\caption[CAN-Shield-Bottom-Layer]{CAN-Shield-Bottom-Layer}
\label{fig:CAN-Shield-Layout-Bottom}
\end{figure}
